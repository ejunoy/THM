\documentclass[12pt]{report}
\usepackage{graphicx}

\title{Jr Penetration Tester}
\author{Emilio Junoy de Juambelz}
\date{\today}

\begin{document}

\chapter{Network security}

\section{Protocols and servers}

\subsection{Telnet}
El protocolo Telnet es un protocolo de la capa de aplicación
usado para conectarse a una terminal virtual u otra computadora.
Usando Telnet, un usuario puede conectarse a otra computadora y 
acceder a su terminal (console) para correr programas, empezar
procesos y realizar tareas del administrador de forma remota.\\

El protocolo Telnet es relativamente sencillo. Cuando un usario 
se conecta, se le pregunta por un nombre de usuario y una 
contraseña. Una vez que el usuario fue autorizado, tendrá acceso
a una terminal remota del sistema. Desafortunadamente, toda esta 
comunicación entre el cliente Telnet y el servidor Telnet no
está encriptada, lo que lo hace un objetivo fácil par los hackers.\\

Un servidor Telnet usa el protocolo Telnet para escuchar conexiones
entrantes en el puerto 23. Consideremos un ejemplo:\\
Un usuario se está conectando a $\textit{telnetd}$, un servidor
Telnet. Los pasos son como siguen:
\begin{enumerate}
  \item Primero, se le pide un nombre de usuario.
  \item Después, se le pide la contraseña (no se muestra).
  \item Una vez que inicia sesión es bienvenido con un mensaje.
  \item El servidor le da una terminal. El "\$" indica que no es
    una terminal root.
\end{enumerate}
Aunque telnet nos dio acceso a una terminal en poco tiempo, no
es un protocolo confiable para administración remota, pues todos
los datos son mandados en texto claro.\\
Telnet no es considerado una opción segura, especialmente porque
cualquiera que esté capturando el tráfico de internet sería capaz
de descubrir el nombre de usuario y la contraseña, lo que le daría
acceso al sistema remoto. La alternativa segura es SSH.

\subsection{Hypertext Transfer Protocol (HTTP) }
El Hypertext Transfer Protocol (HTTP) es el protocolo usado
para transferir páginas web. Tu navegador web se conecta 
al servidor web y usa HTTP para pedir páginas HTML e imágenes, 
mandar forms y subir varios archivos. 
Cada vez que buscamos en la World Wide Web (WWW) usamos el protocolo
HTTP.\\
HTTP manda y recibe los datos en texto claro, entonces podemos
usar una herramienta simple como Telenet (o Netcat) para 
comunicarnos con un servidor web y que actúe como un 
navegador web. La diferencia fundamental es que necesitamos
introducir los comandos relacionados a HTTP en lugar de que el 
navegador lo haga por nosotros.\\
En el siguiente ejemplo, veremos como podemos solicitar una página
de un servidor, más aún, descubriremos la versión del servidor web.
Para conseguir esto, usaremos el cliente Telnet. Lo usamos
porque Telnet es un protocolo simple, además, usa texto claro 
para la comunicación. Usaremos Telnet en lugar de un buscador web 
para pedir un archivo del servidor web. Los pasos son los sigueintes:
\begin{enumerate}
  \item Primero, nos conectamos al puerto 80 usando $\textit{telnet MACHINEIP 80}$
  \item Después, debemos escribir $\textit{GET /index.html HTTP/1.1}$ para 
    obtener la página index.html o $\textit{GET / HTTP/1.1}$ para 
    obtener la página por defecto.
  \item Finalmente, debemos proveer un valor para el host, como $\textit{host: telnet}$ 
    y picar la tecla enter dos veces.
\end{enumerate}
Necesitamos un servidor HTTP (webserver) y un cliente HTTP (web browser)
para usar el protocolo HTTP. El servidor web va a "sevir" un conjunto 
específico de archivos al web browser que pide los recursos.\\
Tres elecciones populares para servidores HTTP son:
\begin{itemize}
  \item Apache
  \item Internet Information Services (IIS)
  \item nginx
\end{itemize}
Apache y nginx son gratis y de código abierto. IIS es de código cerrado
y requiere una licencia.\\

\subsection{File Transfer Protocol (FTP)}



\end{document}
